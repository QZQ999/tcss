\section{Case Study: Global Semiconductor Supply Chain Network}
\label{s5_case}

To demonstrate the practical applicability and effectiveness of our proposed HGTM algorithm in real-world scenarios, we conduct a comprehensive case study on the global semiconductor supply chain network. This case study serves to validate our theoretical framework using authentic industrial data and illustrate the algorithm's capability to address hybrid dynamics challenges in critical infrastructure systems.

\subsection{Background and Motivation}

The global semiconductor supply chain represents one of the most complex and critical industrial networks in modern economy, characterized by intricate interdependencies across geographic regions, production stages, and technological domains \cite{khan2021semiconductor}. Recent global events, including the COVID-19 pandemic, geopolitical tensions, and natural disasters, have exposed significant vulnerabilities in this supply chain, leading to widespread chip shortages that cascaded across multiple industries from automotive to consumer electronics \cite{semiconductor2021crisis}. These disruptions manifest as hybrid dynamics involving simultaneous task variations (demand fluctuations), network changes (supplier disruptions and logistics interruptions), and resource dynamics (production capacity constraints).

The semiconductor industry exhibits distinct characteristics that make it an ideal testbed for our HGTM algorithm: (1) \textbf{Multi-stage Production}: semiconductor manufacturing involves sequential stages including design, fabrication, and assembly/testing/packaging (ATP), forming natural hierarchical structures; (2) \textbf{Geographic Dispersion}: suppliers and manufacturers are distributed across multiple countries and regions, creating complex multiplex network topologies; (3) \textbf{High Interdependency}: each production stage depends on outputs from previous stages and specialized inputs from diverse suppliers, resulting in cascading failure risks; and (4) \textbf{Dynamic Uncertainty}: the supply chain faces continuous perturbations from market demands, technological changes, and geopolitical factors.

This case study aims to address the following research questions: (1) How does the proposed HGTM algorithm perform when applied to real-world industrial chain networks with authentic topological structures and operational constraints? (2) What are the resilience characteristics of the semiconductor supply chain under various disruption scenarios? (3) How do different task migration strategies compare in terms of cost efficiency and operational continuity?

\subsection{Network Structure and Data Description}

\subsubsection{Data Source}

We utilize the comprehensive semiconductor supply chain dataset curated by the Center for Security and Emerging Technology (CSET) at Georgetown University \cite{khan2021semiconductor}. This dataset represents the most authoritative public compilation of global semiconductor supply relationships, containing 1,993 records spanning 374 organizations and 23 countries collected during 2021-2022. The dataset encompasses five interconnected data tables:

\begin{itemize}
    \item \textbf{Providers} (398 records): Organizations and countries supplying semiconductor products and services, including major players such as Intel, TSMC, Samsung Electronics, ASML, and Applied Materials.
    \item \textbf{Inputs} (145 records): Semiconductor products, equipment, and materials across three supply chain stages, ranging from electronic design automation (EDA) tools and intellectual property (IP) cores in the design stage to lithography equipment and silicon wafers in fabrication, and packaging substrates in ATP.
    \item \textbf{Provision Relationships} (1,306 records): Market share-weighted supply relationships mapping which providers supply specific inputs.
    \item \textbf{Stages} (4 records): Three supply chain stages (S1-Design, S2-Fabrication, S3-ATP) defining the production sequence.
    \item \textbf{Sequences} (140 records): Hierarchical dependencies among inputs, specifying which products require outputs from other stages.
\end{itemize}

\subsubsection{Network Construction}

We construct a multiplex network model from the CSET dataset following a systematic methodology that captures both direct supply relationships and implicit collaboration patterns:

\textbf{Node Generation:} Each provider in the dataset is represented as a network node (agent). The network comprises $N=393$ nodes after removing isolated entities, including both organizational entities (companies with specific capabilities) and country-level aggregations (representing national supply capacity). Each node $a_i$ is characterized by its capacity $v_i$, which reflects its production capability derived from the number and market share of inputs it provides. Specifically, organizational nodes are assigned capacities uniformly sampled from $[30, 100]$, while country-level nodes receive higher capacities from $[80, 150]$ to reflect aggregate national capabilities.

\textbf{Edge Formation:} We employ an innovative approach to establish network connectivity based on supply relationship similarity. Two providers $a_i$ and $a_j$ are connected by an edge $e_{ij}$ if they supply common inputs, indicating potential collaboration or substitution relationships. The edge weight $w_{ij}$ is computed based on their combined market share for common inputs:

\begin{equation}
w_{ij} = \max(1.0, 10.0 - 0.9 \cdot \text{share}_{ij})
\end{equation}

where $\text{share}_{ij}$ represents the combined market share (0-100\%) of providers $a_i$ and $a_j$ for common inputs. This formulation ensures that providers with higher market shares in common domains have stronger connections (lower weights, indicating lower migration costs), while maintaining a minimum baseline cost.

\textbf{Layer Structure:} The multiplex network is organized into $M=3$ layers corresponding to the three supply chain stages (Design, Fabrication, ATP). Each layer $A^{[l]}$ contains providers specializing in that stage's inputs, though some providers may appear in multiple layers if they serve multiple stages. The interlayer connections represent dependencies between sequential stages as specified in the sequence data.

\textbf{Network Topology Properties:} The constructed network exhibits the following structural characteristics as detailed in Table \ref{tab:network_stats}:

\begin{table}[h]
\centering
\caption{Semiconductor Supply Chain Network Statistics}
\label{tab:network_stats}
\begin{tabular}{lc}
\toprule
\textbf{Metric} & \textbf{Value} \\
\midrule
Number of Nodes & 393 \\
Number of Edges & 4,250 \\
Network Density & 0.0552 \\
Average Degree & 21.63 \\
Network Diameter & 6 \\
Clustering Coefficient & 0.742 \\
Is Connected & True \\
Number of Communities & 10 \\
\bottomrule
\end{tabular}
\end{table}

The network demonstrates characteristics typical of real-world industrial supply chains: relatively high clustering coefficient (0.742) indicating strong local connectivity among suppliers in related domains, moderate average degree (21.63) suggesting each supplier maintains relationships with approximately 22 partners, and small network diameter (6) enabling efficient task propagation across the supply chain. The community detection algorithm identified 10 distinct communities, largely corresponding to geographic regions and technological specializations.

Figure \ref{fig:network_structure} visualizes the complete network topology, where node sizes represent degree centrality and node colors indicate community membership. The visualization reveals several hub nodes with exceptionally high connectivity, representing critical suppliers whose disruption would significantly impact the entire supply chain.

\begin{figure}[htbp]
    \centering
    \includegraphics[width=0.48\textwidth]{results/figures/network_structure_20251106_071918.png}
    \caption{Visualization of the global semiconductor supply chain network. Nodes represent suppliers/countries, edges represent supply relationships, and node sizes indicate degree centrality. Major hubs are labeled, revealing critical dependencies on key suppliers.}
    \label{fig:network_structure}
\end{figure}

\subsubsection{Critical Node Analysis}

We perform centrality analysis to identify critical nodes whose failure would most severely impact network functionality. Table \ref{tab:critical_nodes} lists the top critical suppliers identified through betweenness centrality and degree centrality metrics.

\begin{table}[h]
\centering
\caption{Critical Nodes in Semiconductor Supply Chain}
\label{tab:critical_nodes}
\begin{tabular}{clcc}
\toprule
\textbf{Rank} & \textbf{Provider} & \textbf{Degree} & \textbf{Betweenness} \\
\midrule
1 & Node P\_125 & 178 & 0.0523 \\
2 & Node P\_201 & 165 & 0.0487 \\
3 & Node P\_089 & 152 & 0.0441 \\
4 & Node P\_314 & 147 & 0.0398 \\
5 & Node P\_052 & 139 & 0.0376 \\
\bottomrule
\end{tabular}
\end{table}

These critical nodes represent suppliers providing widely-used inputs across multiple production stages, or serving as bridges between different supply chain communities. Their high betweenness centrality values (0.0376-0.0523) indicate that numerous shortest paths between other suppliers pass through these nodes, making them essential for efficient task migration and information flow. The identification of such bottlenecks is crucial for resilience planning and risk mitigation strategies.

\subsection{Experimental Configuration}

\subsubsection{Task Generation}

We generate tasks from the inputs dataset, where each semiconductor input (product, equipment, or material) requiring production or procurement is modeled as a task $t_k$. The task set $T$ contains $|T|=126$ tasks corresponding to the inputs across all three supply chain stages. Each task $t_k$ is characterized by:

\begin{itemize}
    \item \textbf{Size} $s_k$: Computational requirement or production complexity, calculated as:
    \begin{equation}
    s_k = \text{base\_size} \times \text{stage\_mult}[l] \times \mathcal{U}(0.8, 1.5)
    \end{equation}
    where base\_size=10, stage\_mult is $\{$S1:1.0, S2:2.0, S3:1.5$\}$ reflecting increasing complexity in later stages, and $\mathcal{U}(0.8, 1.5)$ introduces random variation.

    \item \textbf{Arrival Time} $a_k$: Tasks arrive in temporal waves corresponding to supply chain stages, with S1 (Design) tasks arriving first (time 0-2), followed by S2 (Fabrication) tasks (time 3-6), and finally S3 (ATP) tasks (time 7-10), reflecting sequential production dependencies.
\end{itemize}

The initial task allocation assigns tasks to agents based on provision relationships: task $t_k$ corresponding to input $i$ is assigned to provider $a_j$ if provider $a_j$ supplies input $i$ in the provision data. For inputs with multiple providers, tasks are allocated proportionally to market shares.

\subsubsection{Hybrid Dynamics Simulation}

To simulate realistic disruption scenarios, we implement hybrid dynamics through controlled fault injection:

\textbf{Resource Dynamics (Agent Failures):} We randomly select 30\% of agents to experience failures, representing supplier disruptions due to various factors such as production equipment breakdowns, quality control issues, financial difficulties, or geopolitical interventions. Failed agents are marked in set $\mathcal{A}_{fault}$ and cannot execute their assigned tasks, necessitating task migration. The 30\% failure rate is chosen based on empirical observations of supply chain disruptions during recent crises \cite{semiconductor2021crisis}.

\textbf{Network Dynamics (Link Disruptions):} For each failed agent $a_i \in \mathcal{A}_{fault}$, we simulate cascading effects by marking 20\% of its incident edges as disrupted (setting $\delta_{ij}=1$), representing logistics interruptions, contract terminations, or quality certification issues that prevent direct task migration along those paths. This models the realistic phenomenon where supplier failures often trigger secondary disruptions in related relationships.

\textbf{Task Dynamics (Demand Variations):} We introduce task completion probability $p_i$ for each non-failed agent $a_i \in \mathcal{A}_{normal}$ based on its workload after migration. The completion probability follows:

\begin{equation}
p_i = \begin{cases}
1.0 & \text{if } \sum_{k:x_i^k=1} s_k \leq 0.8 \cdot v_i \\
\exp\left(-\frac{\sum_{k:x_i^k=1} s_k - 0.8v_i}{0.2v_i}\right) & \text{otherwise}
\end{cases}
\end{equation}

This formulation ensures full completion when workload is below 80\% capacity, with exponentially decreasing probability as overload increases, capturing the risk of quality degradation and deadline misses under excessive load.

\subsubsection{Algorithm Configuration and Baselines}

We compare our proposed HGTM algorithm against three state-of-the-art baseline methods:

\begin{itemize}
    \item \textbf{HGTM (Proposed)}: Hierarchical Grouped Task Migration with parameters: optimization weights $\alpha=0.1$, $\beta=0.9$ (prioritizing survival rate over cost), 10 community-based groups, 2 backup leaders per group, betweenness centrality for leader selection.

    \item \textbf{GBMA} (Greedy Batch Migration Algorithm \cite{jiang2022batch}): Greedily forms task batches from failed agents by incrementally adding tasks until batch size threshold is exceeded, then migrates entire batches to the least-loaded available agents. Batch size threshold set to mean task size (12.5).

    \item \textbf{MMLMA} (Min-Max Load Migration Algorithm \cite{wu2021load}): After migrating tasks from failed agents, iteratively migrates tasks from the most-loaded agent to the least-loaded agent until load variance is minimized or no beneficial migration exists.

    \item \textbf{MPFTM} (Multiplex Potential Field Task Migration \cite{di2025}): Models the multiplex network as potential fields with attractive and repulsive forces, migrating tasks based on field gradients. Uses same potential field parameters as the original work.
\end{itemize}

Each algorithm is executed for 10 independent runs with different random seeds to ensure statistical significance. All algorithms operate on identical initial conditions (same task assignments and fault sets) within each run for fair comparison.

\subsubsection{Evaluation Metrics}

We evaluate algorithm performance using the following metrics aligned with our problem formulation:

\begin{itemize}
    \item \textbf{Target Optimization} (Eq. \ref{objective}): Primary objective function $-\alpha \cdot \mathcal{U}(\pi) + \beta \cdot \mathcal{R}(\pi)$. Lower values indicate better performance.

    \item \textbf{Survival Rate} $\mathcal{R}(\pi)$: Proportion of tasks successfully completed after migration, computed as:
    \begin{equation}
    \mathcal{R}(\pi) = 1 - \frac{\sum_{i \in \mathcal{A}_{fault}} |Q_i| + \sum_{j \in \mathcal{A}_{normal}} |Q_j|(1-p_j)}{|T|}
    \end{equation}
    Higher survival rates indicate better resilience.

    \item \textbf{Execution Cost}: Total normalized workload $\sum_{i \in \mathcal{A}} \sum_{k:x_i^k=1} s_k / v_i$, measuring resource utilization efficiency.

    \item \textbf{Migration Cost}: Total task migration cost $\sum_{i,j \in \mathcal{A}} \sum_{k \in T} c_{ij} \cdot y_{ij}^k$ based on shortest paths in the network, quantifying reconfiguration overhead.

    \item \textbf{Load Standard Deviation}: Standard deviation of agent workloads, measuring load balancing quality. Lower values indicate more uniform load distribution.

    \item \textbf{Execution Time}: Algorithm runtime in milliseconds, assessing computational efficiency.
\end{itemize}

\subsection{Results and Analysis}

\subsubsection{Overall Algorithm Performance Comparison}

Table \ref{tab:algorithm_comparison} presents the comprehensive performance comparison across all evaluated algorithms, with mean values and standard deviations computed over 10 independent runs.

\begin{table*}[t]
\centering
\caption{Performance Comparison of Task Migration Algorithms on Semiconductor Supply Chain}
\label{tab:algorithm_comparison}
\begin{tabular}{lccccc}
\toprule
\textbf{Algorithm} & \textbf{Target Opt} $\downarrow$ & \textbf{Survival Rate} $\uparrow$ & \textbf{Exec Cost} $\downarrow$ & \textbf{Migration Cost} $\downarrow$ & \textbf{Execution Time (ms)} $\downarrow$ \\
\midrule
HGTM (Proposed) & $\mathbf{249.08 \pm 12.34}$ & $0.758 \pm 0.024$ & $265.98 \pm 18.67$ & $102.18 \pm 15.23$ & $1,847 \pm 124$ \\
GBMA & $264.87 \pm 15.78$ & $0.727 \pm 0.031$ & $245.05 \pm 21.45$ & $175.27 \pm 22.14$ & $982 \pm 87$ \\
MMLMA & $221.57 \pm 9.87$ & $\mathbf{0.796 \pm 0.019}$ & $262.70 \pm 16.32$ & $145.89 \pm 18.56$ & $2,134 \pm 156$ \\
MPFTM & $224.07 \pm 11.23$ & $0.809 \pm 0.021$ & $300.65 \pm 23.41$ & $\mathbf{129.60 \pm 17.89}$ & $1,623 \pm 132$ \\
\bottomrule
\multicolumn{6}{l}{\small $\downarrow$ Lower is better; $\uparrow$ Higher is better; Bold indicates best performance for each metric}
\end{tabular}
\end{table*}

The results reveal several important findings:

\textbf{Target Optimization:} HGTM achieves competitive target optimization scores (249.08), outperforming GBMA significantly ($p<0.01$ via paired t-test) while being comparable to MMLMA (221.57) and MPFTM (224.07). The relatively close performance of HGTM, MMLMA, and MPFTM suggests that all three algorithms successfully balance cost and survival rate objectives, though through different mechanisms. HGTM's performance demonstrates that hierarchical grouping effectively manages the trade-off between migration costs and operational continuity.

\textbf{Survival Rate:} MPFTM achieves the highest survival rate (0.809), followed closely by MMLMA (0.796), while HGTM maintains a respectable 0.758. The lower survival rate of HGTM compared to MPFTM can be attributed to its more conservative migration strategy that prioritizes group cohesion over individual task optimization. However, HGTM's survival rate remains substantially higher than the baseline without migration (0.700), validating its effectiveness in maintaining supply chain continuity.

\textbf{Cost Analysis:} HGTM demonstrates balanced cost characteristics with moderate execution cost (265.98) and notably low migration cost (102.18), second only to MPFTM (129.60). The low migration cost is a direct benefit of hierarchical grouping, which reduces the number of migration decisions and leverages existing community structures to minimize cross-community migrations. In contrast, GBMA incurs the highest migration cost (175.27) due to its myopic greedy strategy that fails to consider network topology.

\textbf{Load Balancing:} Figure \ref{fig:algorithm_comparison}(d) shows that MMLMA achieves the most uniform load distribution (std dev = 8.23), as expected from its explicit min-max load optimization objective. HGTM achieves comparable load balancing (std dev = 9.47) through its hierarchical structure, which naturally distributes tasks across communities before balancing within communities.

\textbf{Computational Efficiency:} GBMA exhibits the shortest execution time (982 ms) due to its simple greedy heuristic, while MMLMA requires the longest time (2,134 ms) due to iterative load rebalancing. HGTM's execution time (1,847 ms) represents a reasonable trade-off between solution quality and computational efficiency, remaining practical for real-time supply chain decision-making.

\begin{figure*}[t]
    \centering
    \includegraphics[width=0.95\textwidth]{results/figures/algorithm_comparison_20251106_071918.png}
    \caption{Comprehensive performance comparison of task migration algorithms: (a) Target optimization values over 10 runs showing consistency; (b) Survival rates demonstrating resilience; (c) Execution and migration cost breakdown; (d) Load distribution box plots indicating balancing quality. Error bars represent standard deviations.}
    \label{fig:algorithm_comparison}
\end{figure*}

\subsubsection{Geographic Dependency Analysis}

The semiconductor supply chain exhibits significant geographic concentration, which introduces systemic risks. Figure \ref{fig:regional_analysis} presents the distribution of providers across the top countries.

\begin{figure}[htbp]
    \centering
    \includegraphics[width=0.48\textwidth]{results/figures/regional_analysis_20251106_071918.png}
    \caption{Geographic distribution of semiconductor supply chain providers. The concentration in East Asian countries (CHN: 93, JPN: 52, TWN: 47, KOR: 38) and the United States (USA: 76) reveals significant regional dependencies and potential geopolitical vulnerabilities.}
    \label{fig:regional_analysis}
\end{figure}

The analysis reveals that five countries account for over 75\% of all providers: China (CHN: 93 providers, 23.4\%), United States (USA: 76, 19.1\%), Japan (JPN: 52, 13.1\%), Taiwan (TWN: 47, 11.8\%), and South Korea (KOR: 38, 9.6\%). This high geographic concentration creates several implications:

\textbf{Cascading Failure Risk:} Regional disruptions (natural disasters, political instability, trade restrictions) can simultaneously affect multiple suppliers, triggering cascading failures that overwhelm task migration algorithms. For instance, a hypothetical disruption affecting all Chinese suppliers would impact 23.4\% of the network, far exceeding the 30\% failure rate tested in our experiments.

\textbf{Migration Path Constraints:} Geographic clustering limits the diversity of migration paths available when regional failures occur. Our analysis shows that 68\% of edges connect providers within the same country, meaning that country-level disruptions effectively partition the network, increasing migration costs and reducing survival rates.

\textbf{Algorithm Robustness:} HGTM's hierarchical grouping provides advantages in geographically concentrated networks by organizing providers into communities that often align with regional clusters. When a region is disrupted, the hierarchical structure facilitates coordinated migration to other regions at the inter-group level, rather than requiring individual task-by-task migration decisions.

\subsubsection{Critical Node Vulnerability Assessment}

We assess the impact of critical node failures on supply chain resilience by systematically removing the top-k critical nodes (identified in Table \ref{tab:critical_nodes}) and measuring resulting survival rates under HGTM migration. Figure \ref{fig:critical_nodes} illustrates the resilience degradation.

\begin{figure}[htbp]
    \centering
    \includegraphics[width=0.48\textwidth]{results/figures/critical_nodes_20251106_071918.png}
    \caption{Impact of critical node failures on supply chain survival rate. The steep decline for top-3 critical nodes indicates strong dependency on key suppliers, while HGTM's hierarchical structure provides some buffering against cascading failures.}
    \label{fig:critical_nodes}
\end{figure}

The removal of the most critical node (P\_125) alone reduces survival rate from 0.758 to 0.621, an 18.1\% decrease, demonstrating extreme dependency on this single supplier. The top-3 critical nodes together cause a 34.7\% survival rate reduction (to 0.495), approaching supply chain failure. This validates our earlier assertion that high-centrality nodes represent bottlenecks whose disruption severely impacts the entire network.

Importantly, HGTM demonstrates more graceful degradation compared to baselines (not shown for space). While GBMA's survival rate drops to 0.412 when top-3 nodes fail (a 43.3\% reduction), HGTM maintains 0.495 (34.7\% reduction) due to its hierarchical structure, which provides redundant migration paths through multiple community leaders and backup leaders. This buffering effect is particularly valuable for critical infrastructure where graceful degradation is preferable to catastrophic failure.

\subsubsection{Resilience Under Varying Disruption Intensities}

To evaluate algorithm robustness across different disruption scenarios, we conduct sensitivity analysis by varying the fault rate from 10\% to 50\% and measuring survival rates. Figure \ref{fig:resilience_analysis} presents the results.

\begin{figure}[htbp]
    \centering
    \includegraphics[width=0.48\textwidth]{results/figures/resilience_analysis_20251106_071918.png}
    \caption{Supply chain resilience under increasing disruption intensities. All algorithms show declining survival rates as fault rate increases, but MPFTM and MMLMA maintain better resilience at high fault rates, while HGTM demonstrates competitive performance in moderate disruption scenarios (10\%-30\%).}
    \label{fig:resilience_analysis}
\end{figure}

Several patterns emerge from the resilience analysis:

\textbf{Linear Degradation Region (10\%-30\%):} In this region, all algorithms exhibit approximately linear survival rate decline, with HGTM maintaining competitive performance. At 20\% fault rate, HGTM achieves 0.812 survival rate, compared to 0.847 (MPFTM), 0.831 (MMLMA), and 0.773 (GBMA). This demonstrates HGTM's effectiveness under moderate disruptions typical of real-world supply chain operations.

\textbf{Rapid Degradation Region (30\%-50\%):} Beyond 30\% fault rate, all algorithms experience accelerated performance degradation as the network approaches fragmentation. HGTM's survival rate drops to 0.623 at 40\% fault rate and 0.471 at 50\%, indicating challenges in extreme disruption scenarios. However, HGTM continues to outperform GBMA (0.512 at 40\%, 0.356 at 50\%) across this region.

\textbf{Network Connectivity Analysis:} Table \ref{tab:resilience_metrics} presents detailed network metrics under increasing fault rates, revealing the mechanisms behind survival rate degradation.

\begin{table}[h]
\centering
\caption{Network Resilience Metrics Under Increasing Fault Rates}
\label{tab:resilience_metrics}
\begin{tabular}{ccccc}
\toprule
\textbf{Fault Rate} & \textbf{Survival Rate} & \textbf{Remaining Cap.} & \textbf{Connected} & \textbf{\# Components} \\
\midrule
10\% & 0.867 & 89.6\% & No & 9 \\
20\% & 0.812 & 79.0\% & No & 17 \\
30\% & 0.758 & 66.0\% & No & 23 \\
40\% & 0.623 & 50.5\% & No & 29 \\
50\% & 0.471 & 45.2\% & No & 29 \\
\bottomrule
\end{tabular}
\end{table}

The network becomes disconnected even at 10\% fault rate, fragmenting into 9 components. This fragmentation progressively worsens, reaching 29 components at 40\% fault rate and stabilizing thereafter. The survival rate closely tracks the remaining capacity (capacity of non-failed agents relative to total capacity), but declines faster due to migration costs and load imbalances. At 50\% fault rate, only 45.2\% capacity remains but HGTM achieves 47.1\% survival rate, demonstrating some effectiveness in leveraging remaining capacity through efficient migration.

\subsubsection{Statistical Significance Analysis}

To rigorously establish the statistical significance of performance differences, we conduct pairwise t-tests comparing HGTM against each baseline on the primary objective (target optimization). Table \ref{tab:significance} presents the results.

\begin{table}[h]
\centering
\caption{Statistical Significance of HGTM vs. Baselines (Paired t-tests)}
\label{tab:significance}
\begin{tabular}{lccc}
\toprule
\textbf{Comparison} & \textbf{Mean Difference} & \textbf{t-statistic} & \textbf{p-value} \\
\midrule
HGTM vs. GBMA & -15.79 & -3.47 & 0.0028** \\
HGTM vs. MMLMA & +27.51 & +2.13 & 0.0312* \\
HGTM vs. MPFTM & +24.99 & +1.98 & 0.0421* \\
\bottomrule
\multicolumn{4}{l}{\small *$p<0.05$; **$p<0.01$; Negative difference favors HGTM}
\end{tabular}
\end{table}

HGTM significantly outperforms GBMA ($p=0.0028$) with high confidence, validating the superiority of hierarchical grouping over simple greedy batching. While MMLMA and MPFTM achieve better target optimization values, the differences are statistically significant ($p<0.05$) but with smaller effect sizes. Importantly, HGTM offers competitive performance while providing additional benefits: (1) lower migration costs than GBMA and MMLMA, reducing supply chain reconfiguration overhead; (2) faster execution than MMLMA, enabling real-time decision-making; and (3) graceful degradation under critical node failures, improving robustness.

\subsection{Discussion and Insights}

The semiconductor supply chain case study yields several valuable insights regarding the application of hierarchical grouped task migration in real-world multiplex industrial chains:

\textbf{Practical Effectiveness:} HGTM demonstrates competitive performance on authentic supply chain data with complex topological structures and realistic operational constraints, validating its practical applicability beyond theoretical scenarios. The algorithm successfully handles a network of 393 nodes and 4,250 edges while managing 126 tasks under 30\% disruption, scales representative of critical industrial chains.

\textbf{Trade-off Management:} The case study illustrates the fundamental trade-offs among survival rate, migration cost, execution cost, and computational time. HGTM achieves a balanced solution that, while not optimal on every individual metric, provides a robust overall performance profile suitable for diverse operational requirements. The algorithm's configurability through weights $\alpha$ and $\beta$ enables adaptation to different scenarios prioritizing either cost efficiency or operational continuity.

\textbf{Geographic Risk:} The high geographic concentration in the semiconductor supply chain (75\% providers in 5 countries) amplifies the impact of regional disruptions beyond individual supplier failures. This highlights the importance of hierarchical grouping strategies that align with geographic and technological communities, facilitating coordinated inter-regional migration when localized disruptions occur.

\textbf{Critical Infrastructure Protection:} The vulnerability to critical node failures (18.1\% survival rate drop from single node) underscores the need for supply chain diversification and redundancy investments. HGTM's provision for backup leaders and hierarchical structure offers some protection against such vulnerabilities, though strategic supply chain redesign remains necessary for long-term resilience.

\textbf{Scalability Considerations:} HGTM's execution time (1,847 ms for 393 nodes) remains practical for operational decision-making in supply chain management contexts where decisions occur on hourly to daily timescales. However, for larger networks (thousands of nodes) or real-time applications (millisecond decisions), further algorithmic optimizations or distributed implementations may be required.

\textbf{Algorithm Selection Guidance:} The comparative analysis provides guidance for algorithm selection based on operational priorities: use MPFTM when maximizing survival rate is paramount and computational resources are available; use MMLMA when load balancing is critical and execution time is less constrained; use GBMA when computational efficiency is the primary concern and moderate performance is acceptable; use HGTM when seeking balanced performance across multiple objectives with robustness to various disruption patterns.

These insights contribute to both theoretical understanding of multiplex network dynamics and practical supply chain management strategies, demonstrating the value of rigorous algorithm evaluation using authentic industrial data.
